% Options for packages loaded elsewhere
\PassOptionsToPackage{unicode}{hyperref}
\PassOptionsToPackage{hyphens}{url}
%
\documentclass[
]{book}
\usepackage{lmodern}
\usepackage{amsmath}
\usepackage{ifxetex,ifluatex}
\ifnum 0\ifxetex 1\fi\ifluatex 1\fi=0 % if pdftex
  \usepackage[T1]{fontenc}
  \usepackage[utf8]{inputenc}
  \usepackage{textcomp} % provide euro and other symbols
  \usepackage{amssymb}
\else % if luatex or xetex
  \usepackage{unicode-math}
  \defaultfontfeatures{Scale=MatchLowercase}
  \defaultfontfeatures[\rmfamily]{Ligatures=TeX,Scale=1}
\fi
% Use upquote if available, for straight quotes in verbatim environments
\IfFileExists{upquote.sty}{\usepackage{upquote}}{}
\IfFileExists{microtype.sty}{% use microtype if available
  \usepackage[]{microtype}
  \UseMicrotypeSet[protrusion]{basicmath} % disable protrusion for tt fonts
}{}
\makeatletter
\@ifundefined{KOMAClassName}{% if non-KOMA class
  \IfFileExists{parskip.sty}{%
    \usepackage{parskip}
  }{% else
    \setlength{\parindent}{0pt}
    \setlength{\parskip}{6pt plus 2pt minus 1pt}}
}{% if KOMA class
  \KOMAoptions{parskip=half}}
\makeatother
\usepackage{xcolor}
\IfFileExists{xurl.sty}{\usepackage{xurl}}{} % add URL line breaks if available
\IfFileExists{bookmark.sty}{\usepackage{bookmark}}{\usepackage{hyperref}}
\hypersetup{
  pdftitle={Статистика чрез R},
  pdfauthor={Борис Велков},
  hidelinks,
  pdfcreator={LaTeX via pandoc}}
\urlstyle{same} % disable monospaced font for URLs
\usepackage{longtable,booktabs}
\usepackage{calc} % for calculating minipage widths
% Correct order of tables after \paragraph or \subparagraph
\usepackage{etoolbox}
\makeatletter
\patchcmd\longtable{\par}{\if@noskipsec\mbox{}\fi\par}{}{}
\makeatother
% Allow footnotes in longtable head/foot
\IfFileExists{footnotehyper.sty}{\usepackage{footnotehyper}}{\usepackage{footnote}}
\makesavenoteenv{longtable}
\usepackage{graphicx}
\makeatletter
\def\maxwidth{\ifdim\Gin@nat@width>\linewidth\linewidth\else\Gin@nat@width\fi}
\def\maxheight{\ifdim\Gin@nat@height>\textheight\textheight\else\Gin@nat@height\fi}
\makeatother
% Scale images if necessary, so that they will not overflow the page
% margins by default, and it is still possible to overwrite the defaults
% using explicit options in \includegraphics[width, height, ...]{}
\setkeys{Gin}{width=\maxwidth,height=\maxheight,keepaspectratio}
% Set default figure placement to htbp
\makeatletter
\def\fps@figure{htbp}
\makeatother
\setlength{\emergencystretch}{3em} % prevent overfull lines
\providecommand{\tightlist}{%
  \setlength{\itemsep}{0pt}\setlength{\parskip}{0pt}}
\setcounter{secnumdepth}{5}
\usepackage{booktabs}
\ifluatex
  \usepackage{selnolig}  % disable illegal ligatures
\fi
\usepackage[]{natbib}
\bibliographystyle{apalike}

\title{Статистика чрез R}
\author{Борис Велков}
\date{2020-11-19}

\begin{document}
\maketitle

{
\setcounter{tocdepth}{1}
\tableofcontents
}
\hypertarget{ux432ux44aux432ux435ux434ux435ux43dux438ux435}{%
\chapter{Въведение}\label{ux432ux44aux432ux435ux434ux435ux43dux438ux435}}

Целта на Ръководството е да бъде в помощ на статистици, иконометрици и анализатори при работата с R. Ръководството представя полезни техники за събиране, представяне и разпространение на статистически данни и няма за цел да запознае читателя с основите, синтаксиса или структурите на езика. В Интернет могат да се намерят много материали за начинаещи, като добър старт дават например страницата \href{https://support.rstudio.com/hc/en-us/categories/200098757-Learn-R}{Learn R} на Rstudio или книгата \emph{Mastering Software Development in R} \citep{Peng2017}.

Примерите разглеждани в ръководството разроботенни, стартирани и изпълненени на потребителската машина с \href{https://cran.r-project.org/}{R} версия 4.0.3 \href{https://rstudio.com/products/rstudio/}{RStudio} версия 1.3.
За голяма част от използваните в ръководството примери инсталирането на RStudio не е необходимо и би могло да се изпълнят без интегрираната среда за разработка. Поради интеграция на средата с част от използваните в примерите пакети инсталирането на средата е препоръчително и би улеснила създаването и изпълнението на кода.

Пакетите използвани в примерите.

\hypertarget{dataux434ux430ux43dux43dux438}{%
\chapter{Data\textbar Данни}\label{dataux434ux430ux43dux43dux438}}

Some \emph{significant} applications are demonstrated in this chapter.

\hypertarget{ux447ux435ux442ux435ux43dux435-ux43dux430-ux442ux430ux431ux43bux438ux447ux43dux438-ux434ux430ux43dux43dux438}{%
\section{Четене на таблични данни}\label{ux447ux435ux442ux435ux43dux435-ux43dux430-ux442ux430ux431ux43bux438ux447ux43dux438-ux434ux430ux43dux43dux438}}

\hypertarget{ux447ux435ux442ux435ux43dux435-ux43dux430-ux443ux435ux431-ux431ux430ux437ux438ux440ux430ux43dux438-ux434ux430ux43dux43dux438}{%
\section{Четене на уеб базирани данни}\label{ux447ux435ux442ux435ux43dux435-ux43dux430-ux443ux435ux431-ux431ux430ux437ux438ux440ux430ux43dux438-ux434ux430ux43dux43dux438}}

\hypertarget{ux447ux435ux442ux435ux43dux435-ux43dux430-sdmx}{%
\section{Четене на SDMX}\label{ux447ux435ux442ux435ux43dux435-ux43dux430-sdmx}}

\hypertarget{ux447ux435ux442ux435ux43dux435-ux43dux430-xbr}{%
\section{Четене на XBR}\label{ux447ux435ux442ux435ux43dux435-ux43dux430-xbr}}

\hypertarget{ux43eux43fux442ux438ux447ux43dux43eux442ux43e-ux440ux430ux437ux43fux43eux437ux43dux430ux432ux430ux43dux435-ux43dux430-ux441ux438ux43cux432ux43eux43bux438-ocr}{%
\section{Оптичното разпознаване на символи (OCR)}\label{ux43eux43fux442ux438ux447ux43dux43eux442ux43e-ux440ux430ux437ux43fux43eux437ux43dux430ux432ux430ux43dux435-ux43dux430-ux441ux438ux43cux432ux43eux43bux438-ocr}}

\hypertarget{ux438ux437ux432ux43bux438ux447ux430ux43dux435-ux434ux430ux43dux43dux438-ux43eux442-ux443ux435ux431ux441ux430ux439ux442ux43eux432ux435-web-scraping}{%
\section{Извличане данни от уебсайтове (Web Scraping)}\label{ux438ux437ux432ux43bux438ux447ux430ux43dux435-ux434ux430ux43dux43dux438-ux43eux442-ux443ux435ux431ux441ux430ux439ux442ux43eux432ux435-web-scraping}}

\hypertarget{literature}{%
\chapter{Literature}\label{literature}}

Here is a review of existing methods. фффдфсдфсф

\hypertarget{methods}{%
\chapter{Methods}\label{methods}}

We describe our methods in this chapter.

\hypertarget{applications}{%
\chapter{Applications}\label{applications}}

Some \emph{significant} applications are demonstrated in this chapter.

\hypertarget{example-one}{%
\section{Example one}\label{example-one}}

\hypertarget{example-two}{%
\section{Example two}\label{example-two}}

\hypertarget{final-words}{%
\chapter{Final Words}\label{final-words}}

We have finished a nice book.

  \bibliography{book.bib,packages.bib}

\end{document}
